% !Mode:: "TeX:UTF-8"

\chapter{系统详细设计}
\section{服务器端详细设计}
在系统设计过程中,最重要的是根据需求分析及用例模型构建系统静态模型和动态模型。顺序图展示对象之间的交互,这些交互是指在场景或用例的事件流中发生的。协作图是一种交互图,强调的是发送和接收消息的对象之间的组织结构,使用协作图来说明系统的动态情况。状态图说明对象在它的生命期中响应事件所经历的状态序列,以及它们对那些事件的响应。活动图是主要用于业务建模时,用于详述业务用例,描述一项业务的执行过程。设计时,描述操作的流程\cite{zhanghaipan1998}。

\subsection{系统包图}
包图说明了系统各个模块之间的依赖关系,在\ref{sec:serverModelStructure}中我们已经介绍过了系统的模块结构,根据这个结构,本系统的包结构如图\ref{ServerPackage}所示。由于本系统内容比较多,我们这里先给出大概的结构,后面再一一详细描述。
\pic[htb]{包图}{}{ServerPackage}

\subsection{Config Package详细设计}

\pic[htbp]{Config Package类图}{}{ConfigPackage}

\subsubsection{ApplicationContextConfig}
ApplicationContextConfig.java见附录\ref{sec:ApplicationContextConfig}。

Spring框架有两种配置方式,一种是通过XML配置文件进行配置,这种方式将所有的配置信息写入一个指定的XML文件之中,这种方式略显麻烦,在本文中我们采用了另外一种方式,这种方式是利用Java的Annotation机制来进行配置。

系统启动时默认将resources.properties文件\footnote{resources.properties见附录\ref{sec:resources}。}中的键值对初始化成一个Environment实例,我们可以通过getProperty(String): String方法来获得对应的值。

有了\textbf{Environment}实例,我们就可以将配置信息从代码中分离开来。

\subsubsection{WebMVCResource}
WebMVCResource.java见附录\ref{sec:WebMVCResource}。

为了方便进行测试,我们将一些比较特殊的资源从WebMVCConfig.java中独立开来,放到WebMVCResource.java中。这里主要做了两件事情:一个是配置视图解析器,在这里我们设置视图地址的前缀和后缀,方便Controller调用视图。另外一件事就是配置了JSON数据的转换器,用于解析和构建JSON数据,这里我们使用了fastjson\footnote{Fastjson是一个Java语言编写的JSON处理器,由阿里巴巴公司开发。}。

\subsubsection{WebMVCConfig}
WebMVCConfig.java见附录\ref{sec:WebMVCConfig}
这个文件是SpringMVC框架的配置文件,与之前的ApplicationContextConfig类似,这里配置了与Web相关的参数。

\subsection{Database Package详细设计}
\pic[htbp]{Database Package包图}{}{DatabasePackage}

这个包在MVC模型中处于Model层,所有与数据库有关的API都被包含在里面。

\subsubsection{Entity}
\pic[htbp]{Entity Package类图}{}{EntityPackage}

Entity即为实体,对应着MVC模型中的Model,它和数据库中的内容有着直接的一对一映射关系。本系统数据库较为复杂,详细数据库结构图见附录\ref{sec:databasediagram}。这里我们简述一下各个实体的作用,如表\ref{entitytable}所示

\longthreelinetable{entitytable}{Entity表}{2}{ll}
{Entity & 作用\\
}{
Article & 文章的内容和基本信息\\
Code & 用户提交的代码\\
CompileInfo & 代码的编译信息\\
Contest & 比赛的基本信息\\
ContestProblem & 比赛和题目的对应关系\\
ContestTeamInfo & 参赛队伍的信息\\
ContestUser & 比赛的注册用户\\
Department & 学校的部门信息\\
Language & 可以使用的语言以及参数\\
Message & 用户短消息\\
Problem & 题目内容和基本信息\\
ProblemTag & 题目和分类标签的对应关系\\
Status & 代码的评测状态\\
Tag & 分类标签\\
User & 用户信息\\
UserSerialKey & 用户激活码\\
}{
}

\subsubsection{DTO}
\pic[htbp]{DTO Package类图}{}{DTOPackage}

数据传输对象DTO有两种,一种是客户端向服务器传输的数据,一种是模型向上层传输的数据。前者我们通过一个简单的类可以实现。

Hibernate自带的数据库API较为复杂,为了提升效率和简化代码维护成本,我们自己构建了一套用来提取数据库数据的工具。在这类DTO中,我们使用了一个\textbf{@Fields}注解来注明这个DTO的信息来自数据库中的哪些域,然后通过这个field来构建HQL查询语言的\textbf{SELECT}命令。如下所示为UserListDTO.java的部分内容:

\noindent
\ttfamily
\hlstd{}\hllin{001\ }\hlkwc{@Fields}\hlstd{}\hlopt{(\{}\hlstd{}\hlstr{"userId"}\hlstd{}\hlopt{,\ }\hlstd{}\hlstr{"email"}\hlstd{}\hlopt{,\ }\hlstd{}\hlstr{"userName"}\hlstd{}\hlopt{,\ }\hlstd{}\hlstr{"nickName"}\hlstd{}\hlopt{,\ }\hlstd{}\hlstr{"type"}\hlstd{}\hlopt{,\ }\Righttorque\\
\hllin{002\ }\hlstd{}\hlstr{"school"}\hlstd{}\hlopt{,\ }\hlstd{}\hlstr{"motto"}\hlstd{}\hlopt{,\ }\hlstd{}\hlstr{"lastLogin"}\hlstd{}\hlopt{,\ }\hlstd{}\hlstr{"solved"}\hlstd{}\hlopt{,\ }\hlstd{}\hlstr{"tried"}\hlstd{}\hlopt{\})}\\
\hllin{003\ }\hlstd{}\hlkwa{public\ class\ }\hlstd{UserListDTO\ }\hlkwa{implements\ }\hlstd{BaseDTO}\hlopt{$<$}\hlstd{User}\hlopt{$>$\ \{}\\
\hllin{004\ }\hlstd{}\hlstd{\ \ }\hlstd{}\hlslc{//\ Codes}\\
\hllin{005\ }\hlstd{}\hlopt{\}}\hlstd{}\\
\mbox{}
\normalfont
\normalsize


对应生成的HQL语句为\textbf{SELECT userId, email, userName, nickName, type, school, motto, lastLogin, solved, tried FROM User},配合接下来要介绍到的Condition,我们可以组合出基本的HQL查询语句。

在得到这些域后,我们调用对应的EntityDTOBuilder的build方法来得到这些值。

\subsubsection{Condition}
\pic[htbp]{Condition Package类图}{}{ConditionPackage}

我们在本系统中使用Hibernate作为持久层框架,它提供了强大的HQL查询语言,Condition包的主要功能就是提供了Condition组件,它可以翻译成HQL查询语言的where条件,来限定检索范围。

根据实际情况,本系统设计的Condition支持三种条件:
\begin{enumerate}
	\item Order条件:用来限定返回结果的顺序。
	\item PageInfo条件:用来实现返回结果的分页功能。
	\item 普通条件:既Entry,它既可以是一条普通的条件,如\textbf{userId = 5},也可以是一个Condition。在枚举类型ConditionType中,我们定义了许多常用的条件,如等于、不等于、小于、like、属于等等。
\end{enumerate}

对于每个数据库实体类型Entity,都有一个对应的EntityCondition类,如Problem实体有对应的ProblemCondition。这些EntityCondition类都必须继承自BaseCondition类,并且实现它的\textbf{getCondition()}方法。

对于一些比较简单的条件,我们提供了一个\textbf{@Exp}注解,例如在StatusCondition.java中有如下变量:

\noindent
\ttfamily
\hlstd{}\hllin{001\ }\hlcom{/{*}{*}}\\
\hllin{002\ }\hlcom{\ {*}\ Minimal\ status\ id.}\\
\hllin{003\ }\hlcom{\ {*}/}\hlstd{}\\
\hllin{004\ }\hlkwc{@Exp}\hlstd{}\hlopt{(}\hlstd{mapField\ }\hlopt{=\ }\hlstd{}\hlstr{"statusId"}\hlstd{}\hlopt{,\ }\hlstd{type\ }\hlopt{=\ }\hlstd{Condition}\hlopt{.}\hlstd{ConditionType}\hlopt{.}\Righttorque\\
\hllin{005\ }\hlstd{GREATER\textunderscore OR\textunderscore EQUALS}\hlopt{)}\\
\hllin{006\ }\hlstd{}\hlkwa{public\ }\hlstd{Integer\ startId}\hlopt{;}\\
\hllin{007\ }\hlstd{}\\
\hllin{008\ }\hlcom{/{*}{*}}\\
\hllin{009\ }\hlcom{\ {*}\ Submit\ user\ id.}\\
\hllin{010\ }\hlcom{\ {*}/}\hlstd{}\\
\hllin{011\ }\hlkwc{@Exp}\hlstd{}\hlopt{(}\hlstd{mapField\ }\hlopt{=\ }\hlstd{}\hlstr{"userByUserId"}\hlstd{}\hlopt{,\ }\hlstd{type\ }\hlopt{=\ }\hlstd{Condition}\hlopt{.}\Righttorque\\
\hllin{012\ }\hlstd{ConditionType}\hlopt{.}\hlstd{EQUALS}\hlopt{)}\\
\hllin{013\ }\hlstd{}\hlkwa{public\ }\hlstd{Integer\ userId}\hlopt{;}\hlstd{}\\
\mbox{}
\normalfont
\normalsize


如果这两个成员变量不是空,那么最后我们会得到一个形式如同\textbf{WHERE ... userId $>=$ userId and userByUserId $=$ userId ...}的HQL查询语句。

对于一些比较复杂的条件,开发者可以在\textbf{getCondition()}方法中实现复杂的逻辑。

\subsubsection{DAO}
\pic[htbp]{DAO Package类图}{}{DAOPackage}

DAO提供了基础的数据库操作API,例如添加数据、修改、删除、查询等等,通过与DTO和Condition的配合使用,我们可以方便的进行数据库操作,而不需要为每种情况都生成一段冗长的HQL语句。

\subsection{Service Package详细设计}
\pic[htbp]{Service Package类图}{}{ServicePackage}

Service为上层应用提供了一系列特定的数据库操作,根据接口隔离原则\cite{szyperski2002component},我们不希望上层应用直接调用底层的数据库API来进行操作,我们通过Service来隔离它们。在这里,每个EntityService都完成与指定Entity相关的操作,不允许出现跨Entity的调用。

\subsection{Judge Package详细设计}
这个Package包含了与评测器服务相关的内容。

\subsubsection{评测器内核}
评测器内核负责编译、运行、评测用户代码,是一个控制台程序,通过命令行参数来设定评测任务。评测器内核的主函数有如下参数:

%\threelinetable[htbp]{judgecoredescription}{\textwidth}{ll}{Judge Core参数}
\longthreelinetable{judgecoredescription}{Judge Core参数}{2}{ll}
{参数 & 作用\\
}{
-u & 指定任务ID\\
-s & 指定源代码路径\\
-n & 指定题目ID\\
-D & 指定数据文件夹地址\\
-d & 指定运行的工作目录\\
-t & 指定运行时间限制\\
-m & 指定运行内存限制\\
-o & 指定输出大小限制\\
-S & 开启SPJ选项\\
-l & 指定语言类型\\
-I & 指定测试用例的输入文件\\
-O & 指定-I中测试用例的对应标准输出文件\\
-C & 是否需要编译\\
}{
}

评测结束后,它返回三个整数,分别代表评测结果、内存开销、时间开销。

\subsubsection{JudgeService}
\pic[htbp]{Judge Package类图}{}{JudgePackage}
JudgeService在系统启动时开始运行\footnote{见附录\ref{sec:ApplicationContextConfig}},在这个类中我们用队列judgeQueue作为评测器的调度队列。它生成schedulerThread线程用来等待评测任务的到来,它每隔一定的时间间隔(在这里我们设置为3秒)调用StatusService查找所有等待测试的任务,将其标记为OJ\_JUDGING状态,并加入到judgeQueue中。它还配置了若干个JudgeThread线程用来进行多线程评测操作。每个JudgeThread不停的扫描judgeQueue,直到任务的到来,它首先将代码保存至工作目录下,然后构造控制台命令调用\textbf{Runtime.getRuntime().exec(shellCommand)}来和评测器内核交互,并得到结果,然后依据结果来做出相应的更新。


\subsection{Web Package详细设计}